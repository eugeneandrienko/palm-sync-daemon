\documentclass[a4paper,12pt,oneside]{scrartcl}
\usepackage{cmap}
\usepackage[T2A]{fontenc}
\usepackage[utf8]{inputenc}
\usepackage[english]{babel}
\usepackage{textcomp}
\newcommand*{\No}{\textnumero}

\usepackage{indentfirst}

\usepackage[pdftex,
            pdfauthor={Eugene Andrienko},
            pdftitle={palm-sync-daemon description},
            pdfsubject={Technical description of palm-sync-daemon},
            pdfkeywords={Palm, OrgMode, Linux}]{hyperref}
\hypersetup{
    hidelinks,
    unicode=true,
    urlcolor=blue
}
\usepackage{breakurl}

\usepackage[fleqn]{amsmath}
\usepackage{amssymb}
\usepackage{amsfonts}
\usepackage{mathtools}
\usepackage{icomma}

\usepackage{graphicx}
\graphicspath{{images/}}

\usepackage{float}
\usepackage{wrapfig}
\usepackage{afterpage}
\usepackage{paralist}
\usepackage{topcapt}
\usepackage{fancyhdr}
\usepackage{extsizes}

\usepackage{tikz}
\usepackage{pgfplots}
\usepackage{pgfplotstable}

\usepackage{listingsutf8}
\lstset{escapechar=|,
  frame=single,
  showstringspaces=false,
  inputencoding=utf8,
  extendedchars=true,
  captionpos=b,
  breaklines=true,
  postbreak=\raisebox{0ex}[0ex][0ex]{\ensuremath{\hookrightarrow\space}},
  stringstyle=\ttfamily}

\lstdefinelanguage{org}{
  keywords={TODO, VERIFIED, DONE, CANCELLED, SCHEDULED, DEADLINE},
  keywordstyle=\bfseries
}

\usepackage[backend=biber,
    abbreviate=false,
    style=numeric]{biblatex}
\addbibresource{bibliography.bib}


\renewcommand{\le}{\leqslant}
\renewcommand{\ge}{\geqslant}
% hm for break formulas. For ex.: \hm{=}
\newcommand{\hm}[1]{#1\nobreak\discretionary{}%
  {\hbox{$\mathsurround=0pt #1$}}{}}

\renewcommand{\thesection}{\arabic{section}.}
\renewcommand{\theequation}{\thesection\arabic{equation}}
\renewcommand{\thesubsection}{\thesection\arabic{subsection}}
\renewcommand{\labelenumii}{\theenumii}
\renewcommand{\theenumii}{\theenumi.\arabic{enumii}.}

\title{palm-sync-daemon description}
\author{Eugene Andrienko}
%\date{}


%---------------------------------------------------------------------------


\begin{document}

\clearpage
\maketitle
\thispagestyle{empty}
\newpage

\pagestyle{fancy}
\lhead{}
\chead{}
\rhead{palm-sync-daemon}
\lfoot{}
\cfoot{\arabic{page}}
\rfoot{}


\tableofcontents
\listoffigures
\listoftables

\newpage

\begin{tabular}{|l|l|l|}
  \hline
  \textbf{Date} & \textbf{Document version} & \textbf{Changes} \\
  \hline
  \hline
  2024-01-14 & 1.0 & First version \\
  \hline
\end{tabular}

\newpage

%---------------------------------------------------------------------------

\begin{abstract}
  Main purpose of \texttt{palm-sync-daemon} is seamless synchronization between
  Palm and OrgMode files on the Linux PC. Desired goal~--- user just press
  button on the cradle (or cable) and all necessary data synchonize two way
  between Palm and OrgMode files. Without destructive changes on org-files.

  Org-files will be synchronized with the next applications on the Palm:
  \begin{itemize}
  \item Calendar
  \item Memos
  \item Tasks
  \end{itemize}

  Palm device running Palm OS 5.4.7 (Garnet).
\end{abstract}

\section{Desired OrgMode file format}
\label{sec:desired-orgmode-file}

All data from org-files divided to 3 logical parts:
\begin{enumerate}
\item Data for Memos~--- different notes, not belonging to TODO-list of calendar
  data.
\item Data for Tasks~--- tasks from TODO-list
\item Data for Calendar~--- tasks (may be repetitive) from TODO-list with time
  range.
\end{enumerate}

Lets look closer to the each type of data.

\subsection{Notes}
\label{sec:notes}

My notes stored in separate file, for example \texttt{org/notes.org}. Each note
is a first-level headline\footnote{In terms of OrgMode. See
  \url{https://orgmode.org/manual/Headlines.html}}. Under this headline can be
headlines of the next levels, some text of lists~--- all of these belongs to
the this note.

Each note can have categories or do not have categories at all. For example,
categories which I use:
\begin{itemize}
\item \texttt{doings}
\item \texttt{psychotherapy}
\item \texttt{todo}
\end{itemize}

Each note can have priority or TODO keyword\footnote{See
  \url{https://orgmode.org/manual/TODO-Basics.html}} or may not. For example, I
can use next TODO keywords:
\begin{itemize}
\item \texttt{TODO}
\item \texttt{VERIFIED}
\item \texttt{DONE}
\item \texttt{CANCELLED}
\end{itemize}

Notes example:
\begin{lstlisting}[language=org]
  * Camera wrist strap
  * Knots:
  - straight
  - figure eight
  - bowline
  - Austrian conductor
  ** From Ashley Book of Knots:
  - Knot #1119
  - Knot #5
\end{lstlisting}

\subsection{TODO-list tasks}
\label{sec:todo-list-tasks}

TODO-lists stored in a separate file. For example: \texttt{org/todo.org}. Each
TODO-list item is a first-level headline, which starts from TODO keyword
(described above).

Each TODO-list item can have a special tag, which marks item as:
\begin{itemize}
\item Important and urgent~--- \texttt{important\_urgent} tag
\item Important and not urgent~--- \texttt{important\_nonurgent} tag
\item Not important and urgent~--- \texttt{nonimportant\_urgent} tag
\item Not important and not urgent~--- \texttt{nonimportant\_nonurgent} tag.
\end{itemize}

Also, item may not have any tags at all.

Example:
\begin{lstlisting}[language=org]
  * TODO Buy bicycle parts               :important_urgent:
  * TODO Fix printer paper feed          :important_nonurgent:
\end{lstlisting}

Each TODO-list item \textit{may have} property \texttt{SCHEDULED} or
\texttt{DEADLINE} or just an explanatory text right after first-level headline:
\begin{lstlisting}[language=org]
  * TODO Recall if parcel is lost
  SCHEDULED: <2024-01-02 Tue>
  Date of first call: 25-12-2023
  8-800-234-48-99
\end{lstlisting}

\subsection{Calendar tasks}
\label{sec:calendar-tasks}

Calendar tasks stored in the same file as TODO-list tasks (see
section~\ref{sec:todo-list-tasks}). But calendar tasks differs from TODO-list
tasks. Calendar tasks has a time range. Also a repeater interval may be
set\footnote{See \url{https://orgmode.org/manual/Timestamps.html}}.

For example:
\begin{lstlisting}[language=org]
  * TODO Therapy
  SCHEDULED: <2024-01-08 Mon 08:00-09:30 +1w>
\end{lstlisting}

\section{Synchronization scenario~--- first look}
\label{sec:synchr-scen-first-look}

Primarily, I want to create simple but robust synchronization algorithm. It
should neither duplicate nor delete data on the computer or on the Palm. Of
course, data should not be corrupted too easy.

As you see from examples above (see section~\ref{sec:desired-orgmode-file}), I
do not use any special IDs (like UUID) to distinguish records in
org-files. Also, I do not save date of creation or modification of headline
inside drawer~--- I want my files to be visually clean.

Records from Palm and from org-file will be equal if text from Palm record
header and from first-level headline will be equal. If there are no special
headers, like in Memos application~--- then we treat as header all symbols in
note from first symbol to symbol(s) of carriage return (plus line feed).

If user changed the same record in Palm and in org-file
\textit{simultaneously}~--- that's \textbf{his problem}. In that case his record
will be corrupted. If user deleted something in Palm~"--- corresponding record
will not be deleted on the org-file, to preserve org-files contents. \textbf{Do
  not delete anything on Palm~--- do this action only on PC}.

Other org-files content which should be synced properly:
\begin{description}
\item[Tags] should be mapped to categories on the Palm. Handheld device has some
  limitations on the length of category name. So, for example tag name
  \texttt{important\_urgent} is too long and for Palm it should be mapped to
  \texttt{Imp.\,Urgent} category name. Category names should conform Palm Design
  Guidelines~\cite{PalmDesignGuide}.

  If record from org-file does not have tag~--- it should be mapped to
  \texttt{Unfilled} category on the Palm device.
\item[TODO keywords] should be mapped in the next way: \texttt{TODO} and
  \texttt{VERIFIED} keywords mapped to unchecked check-box on the corresponding
  Palm application. \texttt{DONE} and \texttt{CANCELLED} keywords should be
  mapped to checked check-box.

  When synchronizing notes to the Memos application, all TODO keywords should be
  removed.
\item[Dates] should be mapped one by one when synchronizing to/from the
  corresponding applications on the Palm.
\item[Time ranges] should be mapped one by one when synchronizing to/from the
  corresponding applications on the Palm.
\item[Repeater intervals] should be mapped one by one when synchronizing to/from
  the corresponding applications on the Palm.

  When we change TODO keyword to the \texttt{DONE} or \texttt{CANCELLED} in the
  Emacs OrgMode~--- scheduled date will increment to the corresponding repeater
  interval. To avoid unnecessary rollback of this change after synchronization
  with Palm~--- we should compare date in Palm plus $N$ repeater intervals with
  date in org-file ($N$ is integer). If dates equal~--- do not sync date from
  Palm with date in org-file.

  And vice versa~--- if we synchronize org-files with Palm~--- compare date in
  Palm plus $N$ repeater interval with date in org-file. If dates equal~--- do
  not sync date from org-file with date in Palm.

  If dates not equal (for integer $N$)~--- synchronize date as usual.
\end{description}

If first-level headers have some descriptive text below, or bullet lists, or
next-level headers~--- they should be synced as text into corresponding record
in Palm.

To a first approximation, I see the synchronization scenario as follows. User
comes home, insert Palm into the cradle and press button. After that our daemon
discovers Palm device in the system and immediately starts synchronization
process. This process runs in the background and \textbf{should not} require
user attention.

Algorithm of synchronization process can look like this:
\begin{enumerate}
\item \textbf{Synchronize org-files on the PC with data from Palm}. Assume, that
  user did some \textit{important} changes on the go (change TODO-list item
  status, etc) and we need to sync these changes with PC.

  If user delete something on the Palm~--- \textit{do not sync} these changes
  with PC!  Better to delete all necessary records directly on PC to not
  suddenly damage the user org-files.

  \begin{enumerate}
  \item If note(s) exists on the Memos application and do not exists on the
    org-files~--- sync it with org-files (write it to \texttt{notes.org}, for
    example).

    Use symbols from the start of note to the first newline as text for
    first-level headline in org-file. If there are some text after that
    newline~--- add it as text under the first-level headline.
  \item If there is a note, which exists both on the Palm and on the PC~---
    check is there a text under this note? If this text exists, then compare
    text on Palm and on PC, ignoring the newlines. If texts not equals~--- sync
    text on PC with text from Palm.
  \item If TODO-list item exists on the Tasks application and do not exists in
    the org-files~--- sync it with org-files (write it to \texttt{todo.org},,
    for example).

    If TODO-list item has unchecked checkbox~--- sync it with \texttt{TODO}
    keyword. If item has checked checkbox~--- sync it with \texttt{DONE}
    keyword.

    Treat text from the TODO-list header as first-level headline and all text
    below (if exists) as text below first-level headline.
  \item If TODO-list item exists both on Palm and on PC, then:
    \begin{itemize}
    \item Compare the statuses. If record from Palm is checked as done and has
      \texttt{TODO} or \texttt{VERIFIED} statuses on the PC~--- then change
      status to \texttt{DONE} on the PC. If record from Palm is unchecked and
      has \texttt{DONE} or \texttt{CANCELLED} statuses on the PC~--- then change
      status to \texttt{TODO} on the PC.
    \item Compare categories. If category changed on the Palm~--- change it on
      the PC.
    \item Compare planned time. If time changed on the Palm~--- change it on the
      PC.
    \item Compare description for TODO item in Tasks and text below first-level
      headline in the org-file. If there are not equal (ignore newlines)~---
      copy text from Palm to org-file.
    \end{itemize}
  \item If record in Calendar exists in Palm and not exists on the PC~--- copy
    it to org-files. All record properties~--- date, time range, repeater
    interval and text should be copied to org-file too.
  \item If record in Calendar already exists in the org-file and there is no
    repeater interval~--- sync record in the same manner like TODO-list item
    sync, described above.
  \item If record in Calendar already exists in the org-file and both records
    have repeater intervals~--- check is date in org-file equals with date from
    Calendar. How to compare dates~--- described in previous list in
    \textit{«Repeater intervals»} item.

    If dates and repeater intervals are equal~--- do not sync date and repeater
    interval from Palm with date and repeater interval in org-file. Otherwise
    sync it.

    All other record properties~--- sync them in same manner as described above,
    for TODO-list item. Do not change TODO keywords for records on the PC~---
    Calendar on Palm do not has the checkboxes with same meaning.
  \end{enumerate}
\item \textbf{Synchronize data on the Palm with org-files on the PC}. At this
  point all necessary data from Palm already synced with our org-files. So we
  need only to sync new data from PC with our Palm handheld.

  User can add some new data on the PC\footnote{What's why we do not delete
    records on the PC if there are not exists on the Palm PDA.}, so we need to
  add these changes to the Palm PDA.

  \begin{enumerate}
  \item If we have new note on the PC~--- sync it with Memos application.
  \item If we have new TODO-list item on the PC~--- sync it with Tasks
    application.
  \item If we have new TODO-list item with time range~--- sync it with Calendar
    application.
  \end{enumerate}
\end{enumerate}

% ---------------------------------------------------------------------------

\newpage

\printbibliography[heading=bibintoc]

\end{document}

%%% Local Variables:
%%% mode: latex
%%% TeX-master: t
%%% End:
